\documentclass{article}
\usepackage{graphicx}
\usepackage{setspace}
\usepackage[english]{babel}
\usepackage{tabularx,booktabs,ragged2e}
\usepackage{indentfirst}
\usepackage{listings}
\usepackage{upquote}
\usepackage{amsmath, amssymb}
\DeclareMathOperator{\E}{\mathbb{E}}
\DeclareMathOperator{\Prob}{\mathbb{P}}
\usepackage{svg}
\usepackage{bbm}
\usepackage{bm}
\usepackage{hyperref}


\doublespacing
\newcolumntype{L}{>{\RaggedRight\arraybackslash}X}


\title{
    Inspect the Connection between Cryptocurrency and Stock 
    through Markov Switching Process
}

\author{HSIU-HSUAN, YEH}

\begin{document}
\pagestyle{headings}	
\newpage

\maketitle

\begin{abstract}
The reason for this article choosing the markov switching assumption is that I am curious
about if I can identify the direction of capital flow between the cryptocurrency and stock
markets. This kind of question is always difficult since it's hard to decide which leads
the others. However, we can still somehow have some idea about the relationship of these
markets. The main results of this article is that we can observe the close relation in
the early period which means the switching process of markets are highly overlapped, 
while in the recent year, the switching process starts to be isolated. 
\end{abstract}


\section{Introduction}
  Recently, cryptocurrency has gained a lot of attention. Some are optimistic to
the development of the so-called decentralized finance due to the flexibility and 
the freedom of sharing information. However, in the other side of the "freedom"
is the lack of regularity. Therefore, Some scandal and fraud has taken place, for example, the 
collapse of Terra Luna, bankruptcy of FTX which is the previous second-largest 
cryptocurrency exchange before bankruptcy, to name just a few. It seems that there are large amounts
of to speculate capital flows in the cryptocurrency market imposing the crypto assets
in a high risk of price fluctuation. What's more, due to the tight connection in 
financial market, traditional assets will be influenced in some sense. For example,
if crush happens in cryptocurrency market, there might be a large capital flowing
to the other financial markets such as the stocks and bonds for the sake of avoiding
risk.

  The purpose of this research is to examine the connection between cryptocurrencies 
and stock prices using a Markov switching process. The reason that I assume there
is Markov switching in both market is that there two common type of financial markets 
which is the so called "bear market" and "bull market",
Hence, it seems reasonable to assume there are two hidden states for
both markets. Furthermore, I also curious about if the ordered of changing process 
which the cryptocurrency markets leads the stock markets based on the prior knowledge discussed
above can be observed.

  The structure of this article is as follows: I first examine the characteristics of the return 
of S\&P500 and BTC-USD such as stationarity condition, mean and standard error. 
Basic OLS of autoregressive is considered as well. Afterwards, I combine the autoregressive property and 
markov switching assumption to model the assets return. All the reference code can be found
in my github repo, markov-switch. The link to source code: \href{https://github.com/githubjacky/markov-switching/blob/main/src/model.jl}{model.jl}

\newpage

\begin{figure}[h!]
	\centering
	\includesvg[width=300pt]{res/assests.svg}
	\caption{the historical weekly price of BTC-USD and S\&P500}
	
\end{figure}

\begin{figure}[h!]
	\centering
	\includesvg[width=300pt]{res/assets_price.svg}
	\caption{the historical weekly return of BTC-USD and S\&P500}
	
\end{figure}


\newpage

\section{Data}
Both the S\&P500 amd BTC-USD historical weekly price start from 2018/1/1 to 2023/5/1 
is download from the website of the Yahoo Finance with 278 observations. Due to
the non-stationary issues of the price, I transform the price data to log return.
Demonstrate the plot in Figure 1. After transformation, the mean of the S\&P500 
weekly return is about 0.15047\%and the BTC-USD's is about
0.19189\%. Though BTC-USD seems to have higher return, the sample standard 
deviation differ quite a lot. The former is 0.02835 while the latter is 0.10143. 
It turns out that the BTC-USD is more volatile than the S\&P500, in average and thus, 
suffering from the risk of high return fluctuation.

To have the basic understanding about return data of both S\&P500 and BTC-USD 
I conduct the OLS by assuming the existence of auto correlation for both 
S\&P500 and BTC-USD. In the following section I will use the notation of spx and btc
to specify the weekly return of S\&P500 and BTC-USD.
A thousand miles begin with a single step,
let's first fit AR model and later I will broaden the assumption to more general case which
is the MS-AR model. To decide the proper lag for the AR model, inspecting the autocorrelation
plot is always a good starting point. As shown in figure 2 and 3, we can see that the 
autocorrelation for btc persists much longer comparing to the spx.
However, it's quite difficult to interpret such the long persistence. Therefore, 
I only consider lag period of 5 weeks for spx and 9 weeks, around 2 month for btc
in order to fit the AR model.

\newpage

\begin{figure}[h]
	\centering
	\includesvg[width=260pt]{res/acf_btc_return.svg}
	\caption{auto correlation function of the BTC-USD weekly return}
\end{figure}


\begin{figure}[h]
	\centering
	\includesvg[width=260pt]{res/acf_spx_return.svg}
	\caption{auto correlation function of the S\&P500 weekly return}
\end{figure}

\newpage

Before conducting the OLS, we need to ensure the stationarity condition
holds. The hypothesis test I choose is the Augmented Dickey Fuller test with 
constant deterministic for both situation, lag amount of 5 and 9 period for spx and btc.
The ADF statistic is -7.95658 and -4.60413. Because both reject the nll
hypothesis that there is uni-root, the stationarity requirement is satisfied. The 
following is the specification for the OLS. The reason that I consider the constant 
determinstic is that I first make the assumption, and then decide whether to lose 
it or not. After all, the 0 is just the special case for the constant deterministic.

$$
spx_t = \phi_0 + \Sigma_{i=1}^p \phi_i spx_{t-i} + \epsilon_t
$$


\begin{table}[h!]
\begin{center}
\begin{tabular}{lcccccc} 
     & (1) & (2) & (3)  & (4)  & (5)  \\
    Variables &  AR(1)  &  AR(4), $\beta_0=0$  &  AR(4) &   AR(5), $\beta_0=0$  &  AR(5)\\
    \hline
    constant  &   0.00158  &  &  0.00165  & &  0.00205 \\
     &  (0.00170)  &  & (0.00172)  & & (0.00171) \\
     & & & & & \\
    $spx_{t-1}$  &  -0.087531  &  -0.07609  &  -0.07894  &  -0.09447  & -0.09868   \\
     &  (0.06006)  &  (0.06046)  &  (0.06055)  &  (0.06027)  & (0.06033) \\
     & & & & & \\
    $spx_{t-2}$  &  &  0.04725   &  0.04403  &  0.05334 &  0.04956  \\
      & &  (0.0606)  &  (0.0607)  &  (0.06007)  & (0.06011) \\
      & & & & & \\
    $spx_{t-3}$  &  &  0.01386   &  0.01054  &  0.02231 &  0.01852 \\
      & &  (0.06058)  &  (0.06069)  &  (0.06008)  & (0.06012) \\
      & & & & & \\
    $spx_{t-4}$  &  &  -0.07834  &  -0.08142  &  -0.08272 &  -0.08671   \\
      & &  (0.06037)  &  (0.06047)  &  (0.06)  & (0.06005) \\
      & & & & & \\
    $spx_{t-5}$  &  &  &  &  -0.1086$^{\ast}$  &  -0.11274$^{\ast}$  \\
      & &    &  &  (0.05998)  & (0.06003) \\
    \hline
\end{tabular}
\caption{OLS for spx}
\end{center}
\end{table}

In terms of AR(5) with zero mean, the t statistics for $spx_{t-1}$ is -1.57, 
p-value is 0.1182 which is slightly insignificant in terms of the 10\% significant
level. 

\newpage

$$
btc_t = \phi_0 + \Sigma_{i=1}^p \phi_i btc_{t-i} + \epsilon_t
$$

\begin{table}[h!]
\begin{center}
\begin{tabular}{lcccccc} 
     & (1) & (2) & (3)  & (4)  & (5)  \\
    Variables &  AR(1) &  AR(5), $\beta_0=0$  &  AR(5) &   AR(9), $\beta_0=0$  &  AR(9) \\
    \hline
    constant  &   0.00239  &  &  0.00420  & &  0.00308 \\
     &  (0.00606)  &  & (0.006)  & & (0.00589) \\
     & & & & & \\
    $btc_{t-1}$  &  0.08791  &  0.08510  &  0.08316  &  0.12168$^{\ast\ast}$  & 0.1208$^{\ast}$   \\
     &  (0.05974)  &  (0.06088)  &  (0.061)  &  (0.06156)  & (0.06167) \\
     & & & & & \\
    $btc_{t-2}$  &  &  0.01046   &  0.01031  &  0.01930  &  0.01831  \\
      & &  (0.05965)  &  (0.05972)  &  (0.06079)  & (0.06091) \\
      & & & & & \\
    $btc_{t-3}$  &  &  -0.03189   &  -0.03523  &  -0.05182  &  -0.05277\\
      & &  (0.05951)  &  (0.05958)  &  (0.06082)  & (0.06093) \\
      & & & & & \\
    $btc_{t-4}$  &  &  -0.03249  &  -0.04106  &  0.01843  &  0.01714   \\
      & &  (0.05903)  &  (0.05909)  &  (0.06007)  & (0.0602) \\
      & & & & & \\
    $btc_{t-5}$  &  &  0.08378  &  0.08203  &  0.06693  &  0.06568  \\
      & &  (0.05906)  &  (0.05913)  &  (0.05975)  & (0.05988) \\
      & & & & & \\
    $btc_{t-6}$  &  &  &  &   0.01809  &  0.01736 \\
      & & & &  (0.05847)  & (0.05857) \\
      & & & & & \\
    $btc_{t-7}$  &  &  &  &  0.01323  &  0.01242 \\
      & & & &  (0.05848) & (0.05854) \\
      & & & & & \\
    $btc_{t-8}$  &  &  &  &   0.14785$^{\ast\ast}$  &  0.1475$^{\ast\ast}$ \\
      & & & &  (0.05893)  & (0.05902) \\
      & & & & & \\
      $btc_{t-9}$  &  &  &  &   -0.08995  &  -0.08998 \\
      & & & &  (0.05896)  & (0.05904) \\
    \hline
\end{tabular}
\caption{OLS report for btc}
\end{center}
\end{table}

\newpage

\section{Specification}
Inherit from the previous OLS AR(p) specification, let's add the component of markov 
swithching process and name it as MS-AR(p). I assume there are two kinds of 
hidden state for each period t, 
and for different state, the coefficient differs as well. The variance is considered
to be homogeneous for the reason of simplicity.

\begin{align}
  s_t &\in \{ 0, 1 \} \\
  spx_t &= (\mathbbm{1}(s_t = 0)\phi_0^0 + \mathbbm{1}(s_t = 1)\phi_0^1) + \Sigma_{t=1}^p \phi(s_{t-i})spx_{t-i} +\epsilon_t \\
        &= (1-s_t)\phi_0^0 + s_t\phi_0^1 + \Sigma_{i=1}^p \phi_i(s_t)spx_{t-i} +\epsilon_t \\
  \phi_i(s_t) &= (1-s_t)\phi_i^0 + s_t\phi_i^1 \\
  btc_t &= \phi_0(s_t) + \Sigma_{i=1}^p \phi_i(s_t)btc_{t-i} +\epsilon_t
\end{align}

The transition probability matrix:
$$
P =    
\begin{bmatrix}
 \Prob(s_t=0 | s_{t-1}=0) & \Prob(s_t=1 | s_{t-1}=0)  \\
  \Prob(s_t=0 | s_{t-1}=1) & \Prob(s_t=1 | s_{t-1}=1)
\end{bmatrix}
=\begin{bmatrix}
  p_{00} & p_{01}  \\
  p_{10} & p_{11}
 \end{bmatrix}
$$
$$
\Sigma_{j=0}^1 p_{ij} = 1
$$

My main purpose is to identify the markov-switching property. The standard AR
assumption will have such judge that the results of the estimation is a kind of mixture 
leading to the imprecise estimation.
However, it's quite tricky to estimate MS-AR model some assumptions on the 
initial prediction probability are required. I will address this in next section.
I first assume $\epsilon_t$ follows normal 
distribution to apply Quasi-maximum likelihood estimation. I will examine the assumption
through analysis on residual to see if it make sense.


\section{Estimation}
Denote: $S_T = \{s_T, ..., s_{t+1}, s_t, s_{t-1}, ...s_1\} = \{0, 1\}^T$, 
is the hidden state of the whole period. $Y_T = \{y_T, ..., y_{t+1}, y_t, y_{t-1}, ...y_1\}$ 
which can be the observed spx or btc of the whole period.

In the derivation step of quassi log-likelihood function, there are two sorts of probability:

1. prediction probability: $\Prob(s_t = i | Y_{t-1})$

2. filtering probability: $\Prob(s_t = i | Y_{t})$

The parameters to be estimated: $\theta = ( p_{00}, p_{11}, \sigma_e^2, \{ \phi_i^s | s \in {0, 1} \}_{i=0}^p )'$

$$
L(\theta) = \frac{1}{T} \Sigma_{t=1}^T log f(y_t | Y_{t-1}, s_t, \theta)
$$

where f is the pdf of $\epsilon_t$, and since $s_t$ is known we need to take the expectation
and get the quassi log-likelihood $L'(\theta) = \frac{1}{T} \Sigma_{t=1}^T log f(y_t | Y{_t-1}, \theta)$.

$$  
f(y_t | Y{_t-1}, \theta) = 
    \Prob(s_t=0|Y_{t-1})\Prob(y_t|Y_{t-1}, s_{t}=0; \theta) +
    \Prob(s_t=1|Y_{t-1})\Prob(y_t|Y_{t-1}, s_{t}=1; \theta)
$$
$$
\Prob(s_t = j | Y_{t-1}) = p_{0j} * \Prob(s_{t-1}=0|Y_{t-1}) +
                           p_{1j} * \Prob(s_{t-1}=1|Y_{t-1})
$$
$$
\Prob(s_{t-1}=j|Y_{t-1}) = \frac{\Prob(y_{t-1}|s_{t-1}=j, Y_{t-2}; \theta) * 
                                 \Prob(s_{t-1} = j | Y_{t-2})}{\Prob(y_{t-1}|Y_{t-2}; \theta)}
$$

Through the backward induction, we need to consider the initial condition: $\Prob(s_1 = j | Y_0)$,
and I folllow suggestion from Hamilton (1994) set is as the limiting unconditional
counterpart: the third column of $(A'A)^{-1}A'$
$
A = 
\begin{bmatrix}
  I_{2} - P \\ 1_2' 
\end{bmatrix}
$
where I is the identity matrix and P is the transition matrirx, $1_2 = [1, 1]'$

Tips for coding the quassi log-likelihood function is toset the 
initial condition first, and follow the equation mentioned above sequentially.

After finish the construction of the quassi log-likelihood function, I use the BFGS to optimize the quassi
log-likelihood. Moreover, the parameters $p_{00}, p_{11}$ should in the [0, 1] interval
and $\sigma_{\epsilon}^2$ should greater or equal to 0 so I take advantage of the normal
CDF and exponential function to meet the restrictions. Speaking of asymptotic standard error, 
I first calculate the hessian matrix of the quassi log-likelihood function
evaluating it at MLE, and then apply the delta method to get the correct standard error
of $p_{00}, p_{11}. \sigma_{\epsilon}^2$

when it comes to model selection, AIC and BIC are two types of common criterion.

\begin{align}
    AIC &= T\log(\frac{SSR}{T}) + 2(2(p+1)+3) \\
    BIC &= T\log(\frac{SSR}{T}) + (2(p+1)+3)\log T
\end{align}
where T is the number of observations, SSR is the sum of squared residuals and 
(2(p+1)+3) is the number of estimated parameters. To calculate the sum of squared
residuals, the $\mathbbm{1}(s_t = i)$ should be estimated. My personal selection
is using smoothing probability which is specified by Kim(1994) as the 
estimator. 
\begin{align}
  \hat{\mathbbm{1}}(s_t = 0) &= \Prob(s_t = 0 | Y_T) \\
  \hat{\mathbbm{1}}(s_t = 1) &= \Prob(s_t = 1 | Y_T)
\end{align}

\newpage

The derivation of smoothing probability:

$$
\Prob(s_t = i | Y_T) = \Prob(s_{t+1} = 0 | Y_T) \Prob(s_t = i | s_{t+1} = 0, Y_T) +
\                      \Prob(s_{t+1} = 1 | Y_T) \Prob(s_t = i | s_{t+1} = 1, Y_T)
$$
$$
\Prob(s_t = i | s_{t+1} = j, Y_T) 
    = \frac{\Prob(s_{t+1} = j | s_t = i, Y_t)\Prob(s_t = i | Y_t)}
    {\Prob(s_{t+1} = j | Y_t)}
    = \frac{p_{ij}\Prob(s_t = i | Y_t)}{\Prob(s_{t+1} = j | Y_t)}
$$
$$
\Rightarrow \Prob(s_t = i | Y_T) = \Prob(s_t = i | Y_t)(\frac{p_{i0}\Prob(s_{t+1} = 0 | Y_T)}{\Prob(s_{t+1} = 0 | Y_t)} + 
    \frac{p_{i1}\Prob(s_{t+1} = 1 | Y_T)}{\Prob(s_{t+1} = 1 | Y_t)} )
$$

For coding part, I first get the prediction and filtering probability through the 
quassi maximum likelihood estimator, and the use the filtering probability at T as
the initial point, calculate the remained part backward.

\newpage

\begin{table}[h!]
\begin{center}
  \begin{tabular}{lcccccc}
     & (1) & (2) & (3) \\
    Variablues & AR(1) & AR(4) & AR(5)  \\
    \hline
    $(1-s_t)$constant & 0.00667(0.03181) & 0.00868(0.03273) & 0.00887(0.03098) \\
    \\
    $(s_t$constant & -0.02175(0.12128) & -0.01583(0.06318) & -0.01509(0.05955) \\
    \\
    $(1-s_t)$spx$_{t-1}$ & -0.28243(1.12692) & -0.34289(1.12399) & -0.34124(1.02077)  \\
    \\
    $s_t$spx$_{t-1}$ & 1.03979(4.09697) & 0.7952(2.14944) & 0.77522(1.99865) \\
    \\
    $(1-s_t)$spx$_{t-2}$ & & -0.02216(1.04064) &  -0.03667(1.14148) \\
    \\
    $s_t$spx$_{t-2}$ & & 0.39114(1.73105) & 0.38122(1.56481) \\
    \\
    $(1-s_t)$spx$_{t-3}$ & & -0.13722(1.10954) & -0.11505(1.11394) \\
    \\
    $s_t$spx$_{t-3}$ & & 0.26273(1.58801) & 0.28116(1.55246) \\
    \\
    $(1-s_t)$spx$_{t-4}$ & & -0.22786(1.15642) & -0.26006(1.02341) \\
    \\
    $s_t$spx$_{t-4}$ & & 0.309134(1.74543) & 0.30733(1.87587) \\
    \\
    $(1-s_t)$spx$_{t-5}$ & & & -0.13458(1.02849)  \\
    \\
    $s_t$spx$_{t-5}$ & & & -0.01249(2.17433) \\
    \\
    $p_{00}$ & 0.82308(1.57008) & 0.63659(1.99715) & 0.65189(1.87251) \\
    \\
    $p_{11}$ &  0.12336(1.62135) & 0.18717(1.51741) & 0.2035(1.47203) \\
    \\
    $\sigma_{\epsilon}^2$ & 0.00052(0.00087)  & 0.00043(0.00072) & 0.000421(0.0007)\\
    \\
    AIC & -2139.36473 & -2165.29219 & -2150.59379 \\
    \\
    BIC & -2113.9966 & -2118.32152 & -2096.45171 \\
    \hline
  \end{tabular}
  \caption{normal quassi MLE report for spx}
\end{center}
\end{table}

\newpage


\begin{figure}[h]
	\centering
	\includesvg[width=350pt]{res/spx_normal.svg}
	\caption{AR(4) residuals of spx}
\end{figure}

Because AR(4) has the lowest AIC and BIC, I choose it to do the residuals
analysis to see whether the normal assumption is proper. To test whether the residuals
follow a normal distribution, I decide to use the Jarque-Bera test and the test
statistics is 180.239 which the null hypothesis is rejected. Though the null hypothesis
has been rejected, I still think the normal distribution is not a bad approximation.
In other word, with high possibility that the approximation won't deviate a lot changing
the distribution assumption to student-t.

\newpage

\begin{table}[h!]
  \begin{center}
    \begin{tabular}{lcccccc}
       & (1) & (2) & (3) \\
      Variablues & AR(1) & AR(5) & AR(9)  \\
      \hline
      $(1-s_t)$constant & 0.00239(0.52578) & -0.0851(0.46047) & 0.03227(0.201301) \\
      \\
      $s_t$constant & 0.00239(0.52578) & 0.01678(0.11573) & -0.01984(0.2149) \\
      \\
      $(1-s_t)$btc$_{t-1}$ & 0.08791(3.04313) & 0.68697(3.36902) & 0.05143(1.5145)  \\
      \\
      $s_t$btc$_{t-1}$ & 0.08791(3.04313) & -0.01835(1.07007) & 0.09761(1.91753) \\
      \\
      $(1-s_t)$btc$_{t-2}$ & & 0.68697(5.44482) &  -0.00111(1.60678) \\
      \\
      $s_t$btc$_{t-2}$ & & -0.07558(1.25567) & 0.04457(1.42198) \\
      \\
      $(1-s_t)$btc$_{t-3}$ & & -0.27521(2.69772) & -0.02662(1.73871) \\
      \\
      $s_t$btc$_{t-3}$ & & -0.00082(1.07955) & -0.06516(1.67806) \\
      \\
      $(1-s_t)$btc$_{t-4}$ & & -0.11291(3.5996) &  -0.03589(1.56324) \\
      \\
      $s_t$btc$_{t-4}$ & & -0.03358(1.07279) &  0.07499(1.67469) \\
      \\
      $(1-s_t)$btc$_{t-5}$ & & 0.27714(5.59203) & 0.15949(1.73148)  \\
      \\
      $s_t$btc$_{t-5}$ & & 0.05071(1.32777) & 0.00454(1.45309) \\
      \\
      $(1-s_t)$btc$_{t-6}$ & & & 0.23918(1.78003)  \\
      \\
      $s_t$btc$_{t-6}$ & & & -0.20771(1.62786) \\
      \\
      $(1-s_t)$btc$_{t-7}$ & & & 0.10729(1.69226)  \\
      \\
      $s_t$btc$_{t-7}$ & & & -0.07181(1.35164) \\
      \\
      $(1-s_t)$btc$_{t-8}$ & & & 0.15749(1.88957)  \\
      \\
      $s_t$btc$_{t-8}$ & & &  0.10623(1.32402) \\
      \\
      $(1-s_t)$btc$_{t-9}$ & & & 0.25688(2.27741 )  \\
      \\
      $s_t$btc$_{t-9}$ & & & -0.37145(2.17131) \\
      \hline
    \end{tabular}
    \caption{normal quassi MLE report for btc}
  \end{center}
  \end{table}

\newpage

\begin{table}[h!]
  \begin{center}
    \begin{tabular}{lcccccc}
       & (1) & (2) & (3) \\
      Variablues & AR(1) & AR(5) & AR(9)  \\
      \hline
      $p_{00}$ & 0.5(2470) &  0.05078(3.05193) & 0.57067( 2.15313) \\
      \\
      $p_{11}$ &  0.5(2470) &  0.83393(1.44172) &  0.65415(3.61299) \\
      \\
      $\sigma_{\epsilon}^2$ & 0.01009(0.01427)  &  0.00685(0.01384) & 0.00645(0.0122)\\
      \\
      AIC & -1259.17703 & -1378.71757 & -1369.0413 \\
      \\
      BIC & -1233.80891 & -1324.5755 & -1286.36294 \\
      \hline
    \end{tabular}
    \caption{normal quassi MLE report for btc}
  \end{center}
  \end{table}

  \begin{figure}[h!]
    \centering
    \includesvg[width=350pt]{res/btc_normal.svg}
    \caption{AR(5) residuals of btc}
  \end{figure}

I conduct the Jarque-Bera test on the AR(5) of btx as well, the null hypothesis is
rejected. In this case, student-t distribution may be a better choice. However, since 
the coefficient are all significant and the problem might stems from the small size
of dataset, I didn't give it a shot.

\newpage

\begin{figure}[h!]
  \centering
  \includesvg[width=300pt]{res/smooth_spx.svg}
  \caption{smoothing probabiltiy of spx modeled by AR(4)}
\end{figure}


\begin{figure}[h!]
  \centering
  \includesvg[width=300pt]{res/smooth_btc.svg}
  \caption{smoothing probabiltiy of btc modeled by AR(5)}
\end{figure}

\newpage

\begin{figure}[h!]
  \centering
  \includesvg[width=300pt]{res/compare1.svg}
  \caption{predicted state comparison(2018/02/12 - 2019-11-04)}
\end{figure}
\vspace{2cm}

\begin{figure}[h!]
  \centering
  \includesvg[width=300pt]{res/compare2.svg}
  \caption{predicted state comparison(2019/11/11 - 2021-08-02)}
\end{figure}

\newpage

\begin{figure}[h!]
  \centering
  \includesvg[width=300pt]{res/compare3.svg}
  \caption{predicted state comparison(2021/08/09 - 2023-05-01)}
\end{figure}

\section{Conclusion}

After putting a lot of effort isolating the hidden states, we can observe that
the relationship of changing process in two markets differ quite a lot to the past.
Although technically, I get different state's smoothing probability in each markets, 
there is no actual meaning of "state 0" and "state 1". In fact the "state 0" varies 
a lot between two markets. Hence, I try to find have some insight and decide to 
reverse the state 0 as state 1 in stock markets. That is the pattern we observed from
figure 9 to 11.

\newpage

\begin{thebibliography}{9}
    \bibitem{article} Hamilton (1994) Time Series Analysis
    
    \bibitem{article} Kim (1994) Dynamic linear models with Markov-switching
    
    \end{thebibliography}

\end{document}

